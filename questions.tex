% Created 2019-04-20 Sat 18:54
% Intended LaTeX compiler: pdflatex
\documentclass[11pt]{article}
\usepackage[utf8]{inputenc}
\usepackage[T1]{fontenc}
\usepackage{graphicx}
\usepackage{grffile}
\usepackage{longtable}
\usepackage{wrapfig}
\usepackage{rotating}
\usepackage[normalem]{ulem}
\usepackage{amsmath}
\usepackage{textcomp}
\usepackage{amssymb}
\usepackage{capt-of}
\usepackage{hyperref}
\setcounter{secnumdepth}{3}
\author{Keith Lancaster}
\date{\today}
\title{CIS 4339 Practice Exam 2, S 2019}
\hypersetup{
 pdfauthor={Keith Lancaster},
 pdftitle={CIS 4339 Practice Exam 2, S 2019},
 pdfkeywords={},
 pdfsubject={},
 pdfcreator={Emacs 26.2 (Org mode 9.2.3)}, 
 pdflang={English}}
\begin{document}

\maketitle
Run the command \texttt{rails db:seed} to generate the required data for this exam before beginning the problems.

\section{Problems}
\label{sec:orgb5b90c6}
\begin{enumerate}
\item Modify the application so that a Job has a description
\begin{itemize}
\item The form should be modified to allow the description to be entered, and the
show and index pages modified so that the description is visible.
\end{itemize}
\item Modify the page that shows a single job to have the title "Job" in an h1 tag
at the top of the page
\item Add a div that appears at the top of \emph{every} page that includes a link to the
Job index page
\item Add a method to the job that calculates the total time for the job based
on the total of the time required for the tasks associated with the job.
\begin{itemize}
\item Display the time on the job page \emph{and} on the jobs index page in its own column
\end{itemize}
\item Modify the task model so that an error message is displayed if the task description is not entered
\item Add the \texttt{kaminari} gem to the project
(\url{https://github.com/kaminari/kaminari}) and make the necessary
modifications so that the job index page shows 30 jobs at a time and has a
link to move forward or backward (the default link provided by the gem)
\item Modify the tasks index page so that it shows the job name in the job
column (instead of the job hash)
\item Modify the task form so that is has a dropdown menu for the job. In the
dropdown, the \emph{name} of the job should be shown.
\item Add the Rails admin gem (\url{https://github.com/sferik/rails\_admin}) to the
application using the default url of "/admin"
\end{enumerate}
\end{document}
